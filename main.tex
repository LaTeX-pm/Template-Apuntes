% Tiene dos configuraciones extras: "twoside", para diferenciar las páginas pares de las impares, y "marginnote", para poder activar las notas al margen. También, por parte del paquete todonotes, si se escribe "final" se desactivan los TODO's
\documentclass[marginnote]{notebookFMG}

% ========================= Paquetes =========================

\usepackage{lipsum} % Para meter texto de relleno
\usepackage[Lenny]{fncychap} % Paquete para hacer los encabezados de los capítulos más bonitos. Options: Sonny, Lenny, Glenn, Conny, Rejne, Bjarne, Bjornstrup

\usepackage{silence}
% Disable all warnings issued by latex starting with "You have..." tex.stackexchange.com/questions/186967/you-have-requested-package-include-foo-but-the-package-provides-foo
\WarningFilter{latex}{You have requested package}

\usepackage[color]{pkg/theoremenvFMG} % Carga los entornos de los teoremas. Tiene la opción de mostrar colores en el nombre del teorema, si es que se le entrega la opción de "color".
\usepackage[color]{pkg/marginnotesFMG} % Paquetes para hacer notas al margen. Se pueden desactivar todas las notas quitando la opción "showmn"
\usepackage[]{pkg/todonotesFMG}

%% Datos de la portada
\def\tituloportada {Nuevo Titulo de la Portada}
\def\autordeldocumento {Francisco Muñoz Guajardo}
\def\repositorio {github.com/LaTeX-pm/Template-Apuntes}
\def\fecha {\today}
\def\nombredelcurso {Curso ultra bkn}
\def\codigodelcurso {CO-1234}
\def\nombreuniversidad {Universidad de Chile}
\def\nombrefacultad {Facultad de Ciencias Físicas y Matemáticas}
\def\departamentouniversidad {Departamento de la Universidad}

% Si se desea cambiar la tabla de la portada
\renewcommand{\tablaautor}{
    \begin{tabular}{ll}
        Autor: & \autordeldocumento \\
        Repositorio: & \repositorio \\
        Fecha: & \fecha \\
    \end{tabular}
}

\begin{document}
% Creación de la portada
\portada

% Creación de la tabla de contenidos
\indice

\listoftodos

% Configuraciones para el contenido
\contenidoconfig

\chapter{Capítulo de prueba}
\rc{\lipsum[1-2]\MarginNote{Hola soy un texto que se supone que debe de dar un salto de linea en algún momento}}
\rc{\lipsum[3]\MarginNote{Soy una nota al margen}}

\section{Sección de prueba}

\newpar{
En esta sección se probarán los entornos de los teoremas\FMG[Esto aparecerá en el índice]{Comprar pan}
}

\subsection{Entornos con el estilo \textit{plain}}

\begin{theorem}\namethm[]{Teorema Choro}
hola soy un teorema.
\begin{equation}
    e^{i\pi} + 1 = 0
\end{equation}
\end{theorem}

\begin{lemma}[Lema Choro]
HOLA soy un lema. Aquí va \emphname{un nombre}
\end{lemma}

\begin{corollary}[HOLA soy un corolario]
soy un corolario.
\end{corollary}

\begin{proposition}[Proposición chora]
HoLa Soy una proposición.
\end{proposition}


\subsection{Entornos con el estilo \textit{definition}}

\begin{definition}\namethm[]{EDO no lineal}
una \textit{EDO no lineal} es simplemente una EDO que no es lineal.
\end{definition}

\begin{exercise}[Ejercicio]
Lo anterior es tan trivial que ni siquiera vale como ejercicio.
\end{exercise}

\begin{example}[Inserte un ejemplo aquí]
Hola soy un ejemplo.
\end{example}

\subsection{Entornos con el estilo \textit{remark}}

\begin{remark}[Observación]
Estoy observando el teclado.
\end{remark}

\begin{note}[Nota importante]
Comprar más leche de soja.
\end{note}
\begin{conclusion}[Conclusión final]
En conclusión, me gusta el helado.
\end{conclusion}

\missingfigure{falta una figura}

\subsection{etc...}
\rc{\lipsum[4-7]}



\subsection{Subsección de prueba}
\rc{\lipsum[8]}
\subsection{Otra subsección de prueba}
\rc{\lipsum[9-20]}
\chapter{Otro capítulo de prueba}
\rc{\lipsum[10]}
\section{Hola soy otra sección de prueba}
\rc{\lipsum[11-40]}

HOLA soy un pequeño texto de prueba
\end{document}